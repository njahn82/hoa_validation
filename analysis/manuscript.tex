\documentclass[a4paper,man,floatsintext,longtable,noextraspace,12pt]{apa6}

\usepackage[english]{babel}
\usepackage[utf8x]{inputenc}
\usepackage{amsmath}
\usepackage{graphicx}
\usepackage[colorinlistoftodos]{todonotes}
\usepackage{hyperref}

\usepackage{booktabs}
\usepackage{longtable}
\usepackage{array}
\usepackage{multirow}
\usepackage{wrapfig}
\usepackage{float}
\usepackage{colortbl}
\usepackage{pdflscape}
\usepackage{tabu}
\usepackage{threeparttable}
\usepackage{threeparttablex}
\usepackage[normalem]{ulem}
\usepackage{makecell}
\usepackage{xcolor}
% make captions italic

% number lines
% \usepackage{lineno}
% \linenumbers
            
% bibliography
% definitions for citeproc citations
\NewDocumentCommand\citeproctext{}{}
\NewDocumentCommand\citeproc{mm}{%
  \begingroup\def\citeproctext{#2}\cite{#1}\endgroup}
\makeatletter
 % allow citations to break across lines
 \let\@cite@ofmt\@firstofone
 % avoid brackets around text for \cite:
 \def\@biblabel#1{}
 \def\@cite#1#2{{#1\if@tempswa , #2\fi}}
\makeatother
\newlength{\cslhangindent}
\setlength{\cslhangindent}{1.5em}
\newlength{\csllabelwidth}
\setlength{\csllabelwidth}{3em}
\newenvironment{CSLReferences}[2] % #1 hanging-indent, #2 entry-spacing
  {\begin{list}{}{%
   \setlength{\itemindent}{0pt}
   \setlength{\leftmargin}{0pt}
   \setlength{\parsep}{0pt}
   % turn on hanging indent if param 1 is 1
   \ifodd #1
    \setlength{\leftmargin}{\cslhangindent}
    \setlength{\itemindent}{-1\cslhangindent}
   \fi
   % set entry spacing
   \setlength{\itemsep}{#2\baselineskip}}}
  {\end{list}}
\usepackage{calc}
\newcommand{\CSLBlock}[1]{\hfill\break\parbox[t]{\linewidth}{\strut\ignorespaces#1\strut}}
\newcommand{\CSLLeftMargin}[1]{\parbox[t]{\csllabelwidth}{\strut#1\strut}}
\newcommand{\CSLRightInline}[1]{\parbox[t]{\linewidth - \csllabelwidth}{\strut#1\strut}}
\newcommand{\CSLIndent}[1]{\hspace{\cslhangindent}#1}

% tightlist
\providecommand{\tightlist}{%
  \setlength{\itemsep}{0pt}\setlength{\parskip}{0pt}}

\shorttitle{}

\begin{document}
\thispagestyle{otherpage}

%\maketitle

% Change \title, \date, remove \author:
\begin{large}
\textbf{Replication: Hybrid Open Access in Transformative Agreements}
\end{large}

% Add below \maketitle:
\newcommand{\orcid}{%
  \begingroup\normalfont
  \includegraphics[height=6px]{orcid_logo.png}%
  \endgroup
}
Najko Jahn\textsuperscript{1}\textsuperscript{*} (\orcid{} \href{https://orcid.org/0000-0001-5105-1463}{\color{black}{0000-0001-5105-1463}}) \\

\textsuperscript{1} Göttingen State and University Library, University of Göttingen, Germany. \\

\textsuperscript{*} Correspondence: \href{mailto:najko.jahn@sub.uni-goettingen.de}{\color{black}{najko.jahn@sub.uni-goettingen.de}} 

\section*{Abstract}
{}
{\textbf{Keywords}: hybrid open access, transformative agreements, scholarly publishing, big deals, bibliometrics}

\newpage

% QSS wants numbered sections
\setcounter{secnumdepth}{2}

\section{Introduction}\label{introduction}

This study aims to demonstrate the suitability of open scholarly data
sources for assessing the impact of transformative agreements on hybrid
open access. To achieve this, a replication study was conducted by
comparing results from hoaddata, an openly available and continuously
updated dataset on hybrid open access uptake based on Crossref,
OpenAlex, and the cOAlition S Journal Checker Tool, with the established
bibliometric databases Web of Science and Scopus.

This study focuses on the coverage of hybrid journal portfolios included
in transformative agreements between 2019 and 2023. Special attention is
given to potential differences in open access uptake by country when
comparing first-author affiliation data to corresponding authorships.
This is crucial because the lack of publicly available invoicing data
corresponding to authorships plays an essential role in determining
whether an open-access article is supported through transformative
agreements. Data on corresponding authorships have been available on the
Web of Science and Scopus for much longer than in open databases such as
OpenAlex, where this information is still being roled out at the time of
writing. Because of this weakness, open approaches such as hoaddata and
related research use first-authorship data instead.

By conducting a large-scale comparative analysis, this study aims to

\begin{enumerate}
\def\labelenumi{\arabic{enumi}.}
\tightlist
\item
  Determine the strengths and weaknesses of using open data sources in
  monitoring the impact of transformative agreements on hybrid open
  access publishing.
\item
  Assess the coverage and accuracy of open data sources compared with
  established bibliometric databases.
\item
  Evaluate the reliability of first author affiliation data as a proxy
  for corresponding authorship in the context of open access uptake
  analysis.
\end{enumerate}

\section{Background -- Evidence base to measure the effects of
transformative
agreements}\label{background-evidence-base-to-measure-the-effects-of-transformative-agreements}

\subsection{Anforderungen an das
Monitoring}\label{anforderungen-an-das-monitoring}

\begin{itemize}
\tightlist
\item
  esac guidelines
\item
  gemeinsamkeiten und unterschiede zu apc (listenpreise, tatsächliche
  zahlungen, zentrales invoicing, rabatte, waivers)
\item
  insitutionen covern cas, jedoch kann es zu unterschiedlichen
  verrechnugnsformen führen (antielig mit förderer, splitting
  innerhaklbd er einrichtung)
\end{itemize}

\subsection{Bibliometrische Evidenzen}\label{bibliometrische-evidenzen}

\begin{itemize}
\tightlist
\item
  allgmeeiner uptake
\item
  wachstum apcs
\item
  wachstum verträge (konsortien, forschung)
\item
  konsequenzen
\end{itemize}

\section{Data and methods}\label{data-and-methods}

The aim of this study is to demonstrate the suitability of open
scholarly data sources for assessing the impact of transformative
agreements on hybrid open access. To achieve this, results from
hoaddata, an openly available collection of open research information
regarding hybrid open access, was compared with the established
bibliometric databases Web of Science and Scopus. After describing the
initial data sources used, the necessary pre-processing steps to obtain
eligible articles from transformative agreements using open access
evidence, author roles (first and corresponding) and affiliation data
are presented. Overall, xxxx hybrid journals from xxx agreements that
published at least one open access article between 2019 and 2023. formed
the basis of this study.

\subsection{Data sources}\label{data-sources}

\subsubsection{hoaddata}\label{hoaddata}

hoaddata, developed and maintained by the author, is an R data package
comprising information about the uptake of hybrid open access since 2017
from several openly available data sources. It combines article-level
metadata from Crossref and OpenAlex with transformative agreement
information from the cOAlition S Journal Checker Tool (JCT), which links
journal and institutional data to agreements in the ESAC registry.

More specifically, hoaddata uses Crossref, a DOI registration agency,
for obtaining journal publication volume and open access status through
Creative Commons licence information relative to the published version
(``version of record''). Because of limited affiliation metadata in
Crossref (\url{https://doi.org/10.31222/osf.io/smxe5}), hoaddata sources
first-author affiliations from OpenAlex. While the country in which a
first-author was located was used to aggregate country-level statistics,
Research Organization Registry (ROR) identifiers (ROR-ID) were used in
conjunction with ESAC registry information on the duration of an
agreement was used to estimate whether an article was published under
active transformative agreements. This matching benefited from the
availability of the ROR-IDs in both sources. To improve the matching,
JCT data were enriched to include associated institutions, such as
unviersity hospitals.

hoaddata follows good practices for computational reproducibility using
R. The package, which includes data, code, a test suite and
documentation, is openly available on GitHub. To ensure computational
reproducibility while aggregating the data, a GitHub Actions continuous
integration and delivery (CI/CD) workflow interfaces with the SUB
Göttingen's open scholarly data warehouse based on Google BigQuery,
which provides high-performant programmatic access to monthly snapshots
of Crossref and OpenAlex. The package has been regularly updated since
2022 and the version including the computation log is available on
GitHub.

hoaddata is used as a data basis of the Hybrid Open Access Dashboard, a
data analytics services for library consortia and publishers to track
the uptake of hybrid open access through transformative agreements. It
was also used in bibliometric research (Jahn 2025).

\subsubsection{Web of Science}\label{web-of-science}

Clarivate Analytics' Web of Science (WoS) is a well-established
proprietary bibliometric database consisting of several collections
(Birkle et al., 2020). The collections considered in this study were the
Science Citation Index Expanded (SCIE), the Social Sciences Citation
Index (SSCI) and the Arts \& Humanities Citation Index (AHCI).

The WoS provides important data points for studying spending on open
access: author affiliations and roles, differentiation of journal
articles into document types representing different types of journal
contributions, such as original articles or reviews, and open access
status information derived from OurResearch's Unpaywall, the same
provider as Openalex. However, it lacks information about journals and
articles under transformative agreements.

For programmatic access to article-level data, the database of the
Kompetenznetzwerk Bibliometrie (KB) in Germany is used to access
bibliometric data. The KB processes raw XML data provided by Clarivate
Analytics, which is provided as an in-house PostgreSQL database under a
uniform schema. To support reproducibility, KB maintains annual
snapshots of the database. Accordingly, this study used the annual
snapshot from April 2024, which is considered to cover almost the entire
previous publication year (Schmidt et al., 2024).

\subsubsection{Scopus}\label{scopus}

Elsevier's Scopus, launched in 2004, is another widely used proprietary
bibliometric database for measuring research (Baas et al., 2020).
Similar to the Web of Science, Scopus is selective with regard to the
journals it indexes. However, its coverage is substantially more
extensive than that of the Web of Science Core collection (Singh et al.,
2021; Visser et al., 2021). With detailed metadata about article types,
open access status information derived from Unpaywall, author roles, and
disambiguated affiliations, Scopus also contains important data to
assess open access uptake, although no direct information regarding
transformative agreements was available at the time of the study.

This study used the Scopus annual snapshot of April 2024 as provided by
the KB. The same KB curation effort was applied to the Scopus raw data
as for the Web of Science (Schmidt et al., 2024).

\subsection{Data processing steps}\label{data-processing-steps}

\subsubsection{journal matching}\label{journal-matching}

\subsubsection{authorship records}\label{authorship-records}

\subsubsection{Identifying eligible articles under transformative
agreements}\label{identifying-eligible-articles-under-transformative-agreements}

\subsection{Data records}\label{data-records}

\begin{table}[H]
\centering
\caption{\label{tab:methods_overview_table}Coverage of hybrid journals in transformative agreements 2019-23.}
\centering
\begin{tabular}[t]{lrrr}
\toprule
\textbf{} & \textbf{HOAD} & \textbf{Web of Science} & \textbf{Scopus}\\
\midrule
\addlinespace[0.3em]
\multicolumn{4}{l}{\textbf{Hybrid journal metrics}}\\
\hspace{1em}Active journals & 12,890 & 8,655 & 11,888\\
\hspace{1em}Active journals (core) & 12,888 & 8,655 & 11,878\\
\hspace{1em}Active journals (core) with OA & 11,348 & 8,392 & 11,313\\
\addlinespace[0.3em]
\multicolumn{4}{l}{\textbf{Publication metrics}}\\
\hspace{1em}Total published articles & 9,740,015 & 8,616,053 & 8,117,644\\
\hspace{1em}Core articles & 8,158,425 & 6,708,083 & 7,317,703\\
\addlinespace[0.3em]
\multicolumn{4}{l}{\textbf{Digital Object Identifier (DOI) coverage}}\\
\hspace{1em}Articles with DOI & 9,740,015 & 7,713,796 & 8,105,112\\
\hspace{1em}Core articles with DOI & 8,158,425 & 6,695,661 & 7,314,327\\
\addlinespace[0.3em]
\multicolumn{4}{l}{\textbf{Open Access (OA) metrics}}\\
\hspace{1em}OA articles & 998,699 & 1,112,758 & 974,099\\
\hspace{1em}Core OA articles & 969,817 & 1,019,784 & 922,578\\
\addlinespace[0.3em]
\multicolumn{4}{l}{\textbf{Core articles with affiliation data}}\\
\hspace{1em}First author articles & 7,242,542 & 6,294,855 & 7,232,017\\
\hspace{1em}Corresponding author articles & 5,534,207 & 6,291,441 & 6,898,487\\
\bottomrule
\end{tabular}
\end{table}

\section{results}\label{results}

\section*{discussion}\label{discussion}
\addcontentsline{toc}{section}{discussion}

\phantomsection\label{refs}
\begin{CSLReferences}{1}{0}
\bibitem[\citeproctext]{ref-Baas_2020}
Baas, J., Schotten, M., Plume, A., Côté, G., \& Karimi, R. (2020).
Scopus as a curated, high-quality bibliometric data source for academic
research in quantitative science studies. \emph{Quantitative Science
Studies}, \emph{1}(1), 377--386.
\url{https://doi.org/10.1162/qss_a_00019}

\bibitem[\citeproctext]{ref-Birkle_2020}
Birkle, C., Pendlebury, D. A., Schnell, J., \& Adams, J. (2020). Web of
science as a data source for research on scientific and scholarly
activity. \emph{Quantitative Science Studies}, \emph{1}(1), 363--376.
\url{https://doi.org/10.1162/qss_a_00018}

\bibitem[\citeproctext]{ref-schmidt_2024_13935407}
Schmidt, M., Rimmert, C., Stephen, D., Lenke, C., Donner, P., Gärtner,
S., Taubert, N., Bausenwein, T., \& Stahlschmidt, S. (2024). \emph{The
data infrastructure of the {German Kompetenznetzwerk Bibliometrie}: An
enabling intermediary between raw data and analysis}. Zenodo.
\url{https://doi.org/10.5281/zenodo.13935407}

\bibitem[\citeproctext]{ref-Singh_2021}
Singh, V. K., Singh, P., Karmakar, M., Leta, J., \& Mayr, P. (2021). The
journal coverage of web of science, scopus and dimensions: A comparative
analysis. \emph{Scientometrics}, \emph{126}(6), 5113--5142.
\url{https://doi.org/10.1007/s11192-021-03948-5}

\bibitem[\citeproctext]{ref-Visser_2021}
Visser, M., Eck, N. J. van, \& Waltman, L. (2021). Large-scale
comparison of bibliographic data sources: {Scopus, Web of Science,
Dimensions, Crossref, and Microsoft Academic}. \emph{Quantitative
Science Studies}, \emph{2}(1), 20--41.
\url{https://doi.org/10.1162/qss_a_00112}

\end{CSLReferences}

\end{document}